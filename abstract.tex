\begin{abstract}
Tokenization plays a pivotal role in natural language processing (NLP), shaping how textual data is segmented, interpreted, and processed by language models. Despite the success of subword-based tokenization techniques such as Byte Pair Encoding (BPE) and WordPiece, these methods often fall short in morphologically rich and agglutinative languages due to their reliance on statistical frequency rather than linguistic structure. This paper introduces a linguistically informed hybrid tokenization framework that integrates rule-based morphological analysis with statistical subword segmentation to address these limitations. The proposed approach leverages phonological normalization, root-affix dictionaries, and a novel tokenization algorithm that balances morpheme preservation with vocabulary efficiency. The framework also incorporates special tokens for whitespace and orthographic case, including an \texttt{<uppercase>} token to prevent vocabulary inflation from capitalization. Byte Pair Encoding is integrated to support out-of-vocabulary coverage without compromising morphological coherence. Evaluation on the TR-MMLU benchmark—a large-scale, Turkish-specific NLP benchmark—demonstrates that the proposed tokenizer achieves the highest Turkish Token Percentage (90.29\%) and Pure Token Percentage (85.8\%) among all tested models. Comparative analysis against widely used tokenizers from models such as LLaMA, Gemma, and OpenAI's GPT reveals that the proposed method yields more linguistically meaningful and semantically coherent tokens. A qualitative case study further illustrates improved morpheme segmentation and interpretability in complex Turkish sentences. This work contributes to ongoing efforts to improve tokenizer design through linguistic alignment, offering a practical and extensible solution for enhancing both interpretability and performance in multilingual NLP systems.

\textbf{Keywords:} Tokenization, Morphologically Rich Languages, Morphological Segmentation, Byte Pair Encoding, Turkish NLP, Linguistic Integrity, Low-Resource Languages
\end{abstract}
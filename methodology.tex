\section{Methodology}
\label{sec:methodology}

Traditional NLP models primarily relied on word-level tokenization, where each word was treated as an individual token. However, this approach was inadequate for handling out-of-vocabulary (OOV) words, requiring extensive vocabulary lists that resulted in inefficient memory usage \citep{radford_language_2019}. To address this, subword tokenization methods such as BPE and WordPiece emerged, segmenting rare words into smaller, frequently occurring subunits, thereby improving generalization and reducing OOV occurrences. BPE, originally introduced for data compression \citep{gage_new_1994} and later adapted for NLP by \citet{sennrich_neural_2016}, iteratively merges frequent adjacent character pairs into subword units. Similarly, WordPiece, which was initially developed for speech recognition \citep{schuster_japanese_2012}, follows a comparable iterative merging approach but optimizes token selection using likelihood-based probability maximization.

Morphological complexity presents a significant challenge for NLP tokenization, particularly in agglutinative languages such as Turkish, Hungarian, and Finnish. These languages exhibit a high degree of word inflection, resulting in a vast array of surface forms derived from relatively few lemmas \cite{martins_eurollm_2024}. In Turkish, for instance, the word \textit{anlayabildiklerimizden} (‘from what we were able to understand’) is composed of multiple morphemes: \textit{anla-} (\textsc{understand}) + \textit{-yabil} (\textsc{able}) + \textit{-dik} (\textsc{nominalizer}) + \textit{-ler} (\textsc{plural}) + \textit{-imiz} (\textsc{1pl.poss}) + \textit{-den} (\textsc{ablative}). Standard subword tokenization methods such as Byte Pair Encoding (BPE) and WordPiece often fail to capture such rich internal structures, fragmenting words in ways that obscure grammatical function and semantic interpretation \cite{kaya_effect_2024}. This misalignment reduces linguistic coherence and can negatively impact downstream tasks, highlighting the need for tokenizers that are sensitive to language-specific morphological and phonological features.

\subsection{System Architecture}

At the core of the system lies a hierarchical dataset used to separate words into their morphological components. The first file, \textbf{\texttt{kokler.json}}, hosts noun and verb roots in our language, as well as special control tokens critical for the model's operation such as \texttt{<uppercase>}, \texttt{<unknown>}, \texttt{<pad>}, and \texttt{<eos>}. The second file, \textbf{\texttt{ekler.json}}, hosts Turkish derivational and inflectional suffixes; the most distinguishing feature of this file is that suffixes with the same grammatical function but changing according to sound harmony (e.g., plural suffixes \textit{-lar} and \textit{-ler}) are grouped under a single ID. The third file, \textbf{\texttt{bpe\_tokenler.json}}, holds subword units created with the \textit{Byte-Pair Encoding} algorithm for rare word parts or foreign-origin structures that do not directly match in the root or suffix dictionaries.

\subsection{Encoding (Tokenization) Process}

The \textit{Encoding} phase, where text is converted into numerical data the model can understand, is based on the "Longest Prefix Match" algorithm. The process begins by splitting the raw text according to whitespace characters. At this stage, the positions of all uppercase letters are recorded so that uppercase information in the text is not lost, and compound spellings (CamelCase etc.) are separated into segments according to their internal structures.

For each separated word part, the system scans with a specific priority order: First \textbf{roots}, then \textbf{suffixes}, and finally \textbf{BPE tokens} are checked. The longest matching part found is added to the token list; if no match is provided in any database, the \texttt{<unknown>} tag is assigned to the part. Additionally, the \texttt{<uppercase>} marker is added before segments starting with an uppercase letter in the original text to preserve case information. In the final step, all obtained token objects are converted to their corresponding numerical IDs in the JSON files.

\subsection{Decoding (Text Construction and Sound Events)}

The section of the system with the most complexity and linguistic intelligence is the \textit{Decoding} phase. This process works on the "Single ID, Multiple Views" principle; that is, the model produces functional IDs, not the surface forms of words. The Decoder, while converting these IDs to text, dynamically resolves Turkish sound events by looking both forward (lookahead) and backward (lookbehind).

\textbf{In the analysis of roots}, the system looks forward, i.e., to the token coming after it. The list of roots is ordered from the form of the word most susceptible to change to the least. For example, the root "sıcak" (hot) hosts \texttt{["sıcak", "sıcağ", "sıca"]} variations under a single ID (100). If a suffix starting with a vowel follows this ID (e.g., "-ı"), the system selects the softened form "sıcağ" to form the word "sıcağı". Similarly, if the suffix "-yor" follows the verb "ağla" (cry), the vowel narrowing rule is applied, the "ağlı" form is selected, and the output "ağlıyor" is produced.

\textbf{In the analysis of suffixes}, the system decides by looking backward, i.e., to the token preceding it. Suffixes are shaped according to vowel harmony and consonant hardening rules:

\begin{enumerate}
    \item \textbf{Vowel Harmony:} In suffixes with two variations like the plural suffix, if the last vowel of the previous word is front, "-ler" (meyveler) is selected; if back, "-lar" is selected. In suffixes with four variations (possessive, accusative, derivational suffixes, etc.), the appropriate one from the sounds "ı, i, u, ü" is determined based on whether the previous vowel is unrounded/rounded and front/back. For example, the privative suffix coming after the word "akıl" enters the "ı" harmony and takes the form "akılsız".
    \item \textbf{Consonant Hardening:} Locative (-da/-de), ablative, or copula suffixes harden if the previous word ends with one of the "FıSTıKÇı ŞaHaP" consonants. For example, the ablative suffix coming after the word "ekmek" becomes "-ten" (ekmekten) instead of "-den"; the copula suffix coming after the word "saat" takes the form "saattir".
    \item \textbf{Complex Situations:} Structures like the future tense (-acak/-ecek), the denominal noun-making "-lık" or the diminutive suffix "-cık" look at both the previous token (for vowel/consonant harmony) and the next token (for whether softening will occur) simultaneously. For example, while the "-lik" suffix coming to the word "kalem" normally becomes "kalemlik", when a vowel suffix comes after it, it softens and transforms into the "kalemliği" form; the "-cik" suffix coming to the word "kedi" remains as "kedicik".
\end{enumerate}
